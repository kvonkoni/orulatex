%==============================================
% ORULATEX ASSIGNMENT TEMPLATE
%==============================================

%==============================================
% BEGINNING OF PREAMBLE
%==============================================
\documentclass[12pt,letterpaper]{orulawork}
%==============================================

%==============================================
% ASSIGNMENT AND TEACHER INFORMATION (Update this)
%==============================================
\teacher{Abraham Lincoln}
\assignment{Homework 5}
\coursecode{MATH 101}
\coursename{Introduction to Arithmetic}
\topics{Order of operations, Negative Numbers, Whole Numbers, Fractions}
\duedate{December 15, 1863}
%==============================================

%==============================================
% BEGINNING OF DOCUMENT
%==============================================
\begin{document}
%==============================================

\instructions{Complete all of the following exercises. Show all of your work. Answers alone will not be awarded \emph{any} marks. The number of points awarded for successfully completing each exercise is indicated in the right margin. The assignment is marked out of a total of \pointssum}

\begin{question}{6}
Evaluate the following using order of operations.
\begin{exercises}
\item $6000 - 8 \div 2$
\item $15 - 2 ( 3 - 1) + 8$
\end{exercises}
\end{question}

\begin{question}{1}
Blake will buy his friend's laptop for \$600. He will do this by paying his friend \$12 per week. How many weeks will it take to pay for the laptop?
\end{question}

\begin{question}{3}
Evaluate. Where possible reduce the answer to the lowest terms.
\begin{exercises}
\item $\frac{3}{2}+\frac{1}{8} \times \frac{3}{4}$
\item $(\frac{1}{3}-2\frac{3}{5})(\frac{5}{6}-\frac{1}{3})$
\item $\frac{2}{3}(\frac{1}{6}+\frac{1}{3})$
\end{exercises}
\end{question}

%==============================================
% END OF DOCUMENT
%==============================================
\end{document}
%==============================================